%\documentclass{kuthesis}
\documentclass[twocolumn]{jsarticle}
%\documentclass[12pt]{jsarticle}

\usepackage{bm}

\title{不審者からユーザーを守るロボットの\\動作アルゴリズムの構築}
\author{内藤 謙史郎}
\date{2018/12/21}

\begin{document}
\maketitle
\section{はじめに}
一人で帰る夜道……不審者に遭遇しないかと不安になることも多いだろう。そんなとき、一人で帰るよりも二人・複数人で一緒に帰る方が不審者に襲われにくいということは想像に難くない。しかし、いつも一緒に帰ってくれる人がいるとは限らない。その役割をロボットに任せることはできないだろうか。

現状でも防犯ブザーや催涙スプレーのように不審者から身を守るためのグッズがあり、広く利用されているが、これらは不審者に襲われてしまった後の対応に用いられる面が強い。それらとまた違った方向で、ロボットの存在が前もって犯罪を抑止できるような状況を実現し、ユーザーにより確かな安心感を提供できればと考える。

\section{関連研究}
関連のある他の研究をあげながら、本研究の立ち位置を考える。まず、目的地まで(障害物を避けながら)ナビゲートするロボットに関する研究がある\cite{1}。この研究をベースに、加えて、不審者との遭遇に対応する動作を実装することになる。どのような動作になるかというところで、周囲の脅威を最小化する護衛ロボットの研究\cite{2}や、ロボットの近づき方とそれに対して人間が受ける印象の研究\cite{3}が参考になると思われる。\cite{3}に関しては、人間がやろうとしていることを極力邪魔しない動作を追求する研究であるが、本研究ではそれとは逆に不審者に対して行動を制限させるような、圧力を与えられる動きを模索することになる。

\section{対象と問題の設定}
human(ユーザ), robot, enemy(不審者)という三者を対象として考える。humanは守られる対象である。robotは普段humanと目的地まで連れ立って歩くが、不審者を検知した際にはhumanを守る動きをとるものとする。enemyについては、humanの所有物を奪おうとしている、危害を加えようとしているなど様々な想定が可能であるが、今回はそれらを総括してひとまずhumanのいる地点まで近づくという動きをとるものとする。

以上の三者の中において、ロボットにどのような動きをとらせるのが最適かを考えていく。

\section{研究の流れ}

以下のような流れに沿って、ロボットに実装するべき最適なアルゴリズムを探す。

\subsection{人間の動きを観察}
ロボットにある役割を担わせたい時、どのような動作を行えば良いか考えるにあたり、最初に参考とすべき情報として、人間がその役割をどのように果たすのかということがあげられる。

3名の協力者の方にそれぞれhuman, robot, enemyのロールプレイをお願いし、その様子を動画に撮影した。人間ならば不審者からユーザを守ろうとした時どのような動きをとるかを観察し、アルゴリズムとして落とし込めそうな部分を見つける。

\subsection{シミュレーション}
先の過程で得られたアルゴリズム案を、まずはシミュレーションの中で実行してみる。

シミュレータmorse上に設定したフィールドに対象となるの三者を配置し、humanをキーボードで操作可能なように、enemyをhumanに接近するようにスクリプトにより設定し、それらに対してrobotがとるべきと考えられる動きをシミュレーション上で実装する。

(ユーザの安心感・不審者側の警戒心といったものをシミュレーションでどう測る?)

\subsection{実験室実験}
シミュレーション上で実装したアルゴリズムをRobovie-R3の実機にインストールし、実験室内で動作させたい。

\section{アルゴリズム}
\begin{itemize}
	\item robotはhumanとenemyを結ぶ直線上に移動する。その際、直線上のどの位置に行くべきか(例えばhuman寄り、enemy寄り、真ん中、など)をパラメータ化する。
	\item robotはenemyの方を常に向き続ける。そうすることでenemyに対してプレッシャーを与えることができる。
	\item enemyどの程度(どのように)近づいてきたら防衛の動きに移るかをパラメータ化する。
	\item ロボットの移動速度の上限をパラメータ化する(おそらく実機の移動速度上限に合わせる)。
\end{itemize}


\subsection*{Sanity Check}
\begin{itemize}
	\item $P_R = T_R \Rightarrow \bm{v}_R_{des} = \bm{0}$ \\
    : at equilibrium point, protector stays when enemy keeps still
	\item $|v_e_{\prep} - v_h_{\prep}| \nearrow \Rightarrow \bm{v}_R_{des} \nearrow$ \\
    : protector moves faster to the meeting point when enemy comes faster
	\item Protector always faces enemy \\
    : facing enemy to pressure on enemy
	\item $|v_e_{\parallel} - v_h_{\parallel}| \nearrow \ge v_R_{max} \Rightarrow r_{protect} \searrow$ \\
    : when enemy moves too fast along target, protector would shrink the range of protection
\end{itemize}


\section{実装}

/* TODO */

\section{評価と考察}
提案手法vs障害物を避けるだけのアルゴリズム

シミュレーション上or実機と人間の被験者


\begin{thebibliography}{3}
\bibitem{1} Towards a Socially Acceptable Collision Avoidance for a Mobile Robot Navigating Among Pedestrians Using a Pedestrian Model
\bibitem{2} Controlling the Movement of Robotic Bodyguards for Maximal Physical Protection
\bibitem{3} May I help you? - Design of human-like polite approaching behavior -
\end{thebibliography}
\end{document}
